\documentclass{article}

\usepackage{graphicx}% Include figure files
\usepackage{dcolumn}% Align table columns on decimal point
\usepackage{bm}% bold math
\usepackage{amsmath}
\usepackage{ifthen}
\usepackage{amsfonts}
\usepackage{dsfont}
\usepackage{boxedminipage}
\usepackage{hyperref}

\newcommand{\R}{\ensuremath{\mathds{R}}}
\newcommand{\C}{\ensuremath{\mathds{C}}}
\newcommand{\Q}{\ensuremath{\mathds{Q}}}
\newcommand{\N}{\ensuremath{\mathcal{N}}}
\newcommand{\Z}{\ensuremath{\mathds{Z}}}
\newcommand{\D}{\ensuremath{\mathds{D}}}
\newcommand{\W}{\ensuremath{\mathds{W}}}
\newcommand{\B}{\ensuremath{\mathds{B}}}
\newcommand{\mS}{\ensuremath{\mathds{S}}}
\newcommand{\I}{\ensuremath{I}}
\newcommand{\1}{\ensuremath{\mathds{1}}}
\newcommand{\SO}{\ensuremath{\mathsf{SO}}}
\newcommand{\Sp}{\ensuremath{\mathsf{Sp}}}
\newcommand{\s}{\ensuremath{\mathfrak{s}}}
\newcommand{\n}{\ensuremath{\mathfrak{n}}}
\newcommand{\Ocplx}{\ensuremath{\mathcal{O}}}
\newcommand{\0}{\ensuremath{0}}
\newcommand{\cset}[1]{\mathcal{#1}}

\newcommand{\softname}{SSA Toolbox}

\newcommand{\KLD}{D_{\text{KL}}}
%\DeclareMathOperator*{\KLD}{KL}

\DeclareMathOperator*{\E}{E}
\DeclareMathOperator*{\sign}{sign}
\DeclareMathOperator*{\trace}{tr}
\DeclareMathOperator*{\Var}{Var}
\DeclareMathOperator*{\Cov}{Cov}
\DeclareMathOperator*{\expm}{expm}
\DeclareMathOperator*{\myspan}{span}

\begin{document}

\title{SSA Toolbox \input version.tex \\ Manual}
\author{Jan Saputra M\"uller, Paul von B\"unau, \\ Frank C.~Meinecke, Franz J.~Kir\'{a}ly, Klaus-Robert M\"uller}

\maketitle

\tableofcontents

\newpage

\section{Technical Prerequisites}

\paragraph{Obtaining the latest SSA~Toolbox}

The latest version of the SSA~Toolbox is available from the offical SSA homepage: 
\begin{center}
	\url{http://www.stationary-subspace-analysis.org}
\end{center}
There you can also find pointers to further references, example data and a link 
to the SSA mailing list. 

\paragraph{Platforms}

The SSA~Toolbox is written in the platform-independent Java programming language; 
platform-specific numerical libraries\footnote{BLAS+LAPACK provided through
jblas (see \url{http://www.jblas.org}).} are included for several target architectures. 
The SSA~Toolbox requires the Java Runtime Environment\footnote{See 
\url{http://www.java.com/getjava}} version 1.5 or later. Most operating systems
have a Java Runtime Environment pre-installed, you might be able to find out 
the version by typing \texttt{java -version} on the command line. The SSA~Toolbox has 
been tested on the following platforms.
\begin{itemize}
	\item Microsoft Windows (only 32 bit Java Virtual Machine)
	\item Linux (32 and 64 bit)
	\item Mac OS X (32 and 64 bit)
\end{itemize}
On 64 bit systems you will have to install the package \texttt{libgfortran3}. This can be
done for example under Ubuntu with
\begin{center}
	\texttt{apt-get install libgfortran3}
\end{center}

\paragraph{Installation and Running}

Download the software from the SSA homepage\footnote{\url{http://www.stationary-subspace-analysis.org}}. 
The SSA~Toolbox comes as a single \texttt{.zip} or \texttt{.tar.gz} archive. 
After unpacking, you can start the SSA~Toolbox by opening the file \texttt{ssa.jar}
with the default method, e.g.~by double-clicking on it under Microsoft Windows, 
OS X and some Linux distributions.

You can also manually invoke the SSA~Toolbox by typing
\begin{center}
  \texttt{java -jar ssa.jar}
\end{center}
on the command line or running the shell scripts \texttt{ssa.sh} (Linux)
or \texttt{ssa.bat} (Windows). If you want to use the SSA~Toolbox directly from Matlab, 
you can use the wrapper script \texttt{ssa.m}. Type \texttt{help ssa} on the Matlab 
command line to find out about the format of its input and output parameters. 
Note that if you invoke the SSA Toolbox from within Matlab, it will use its
internal JVM unless you specify an external JVM, e.g. using the environment
variable \texttt{MATLAB\_JAVA} under Linux.

\section{The SSA Model and Terminology}

Stationary Subspace Analysis \cite{PRL:SSA:2009} factorizes a multivariate time-series 
into its stationary and non-stationary components. That is, we assume that the data 
generating system consists of $d$ stationary source signals 
${\mathbf s^\s}(t) = [s_1(t), \dots, s_d(t)]^\top$ and $D-d$ non-stationary source signals
${\mathbf s^\n}(t) = [s_{d+1}(t), \dots, s_D(t)]^\top$ and that the observed signals 
$x(t)$ are a linear superposition of these sources,
\begin{equation}
  {\mathbf x}(t) = A {\mathbf s}(t) = 
    \begin{bmatrix} A^{\s} & A^{\n} \end{bmatrix}
    \begin{bmatrix} {\mathbf s^{\s}}(t) \\ {\mathbf s^{\n}}(t) \end{bmatrix}
\label{eq:mixing_model}
\end{equation}
where $A$ is an invertible matrix. Note that in contrast to 
Independent Component Analyis~\cite{ICABook} we do \textit{not} assume that 
the sources ${\mathbf s}(t)$ are independent. We refer to the spaces spannend 
by the columns of $A^{\s}$ and $A^{\n}$ as the stationary ($\s$-) and 
non-stationary ($\n$-) space respectively.  
%Please note that for the purpose
%of the SSA~Toolbox, we have interchanged the dimensions, i.e.~contrary
%to the publications \cite{PRL:SSA:2009}, the time runs along the first dimension
%and the different sources and signals each correspond to a 
%column\footnote{This is to be consistent with the majority of statistics and signal
%processing software packages, in particular to facilitate easy data exchange
%via the CSV file format.}

The SSA algorithm factorizes the observed signals $x(t)$ according to
Equation~\ref{eq:mixing_model}, i.e.~it finds a linear transformation
\begin{equation}
\label{eq:est_demixing}
  \hat{A}^{-1} = \begin{bmatrix} \hat{P}^{\s} \\ \hat{P}^{\n} \end{bmatrix}
\end{equation}
that separates the \s-sources from the \n-sources. The inverse of the estimated 
demixing matrix $\hat{A}^{-1}$ is the estimated mixing matrix, 
\begin{equation}
\label{eq:est_mixing}
 \hat{A} = \begin{bmatrix} \hat{A}^{\s} & \hat{A}^{\n} \end{bmatrix}, 
\end{equation}
and the estimated stationary and non-stationary sources are thus given by
\begin{align}
\label{eq:est_s_sources}
  \hat{\mathbf s}^{\s}(t) & = \hat{P}^{\s} {\mathbf x}(t) \\
\label{eq:est_n_sources}
  \hat{\mathbf s}^{\n}(t) & = \hat{P}^{\n} {\mathbf x}(t) 
\end{align}
respectively.
Note that SSA only identifies the $\s$-projection and the $\n$-space uniquely.
$\hat P^{\n}$ is chosen such that it is orthogonal to $\hat P^{\s}$.
Therefore, $\hat A^{\s}$ is orthogonal to $\hat A^{\n}$.\\
The SSA Toolbox allows for input and output in two formats: the first
dimension corresponding to the channels and the second to time or vice versa.  
Note that the above definitions holds for the former case, 
i.e.~$\text{channels} \times \text{time}$. If the other output format has
been chosen, then the results (Equations~\ref{eq:est_demixing}, \ref{eq:est_mixing}, 
\ref{eq:est_s_sources} and \ref{eq:est_n_sources}) are transposed.

\section{Input}

The input to the SSA Toolbox consists of
\begin{itemize}
 \item Data: the time series $x(t)$, either $\text{channels} \times \text{time}$ or 
				$\text{channels} \times \text{time}$.

 \item Segmentation of the time series into epochs, either
	\begin{itemize}
	  \item equally-sized (segmented automatically by the SSA Toolbox); or 
	  \item according to a user-supplied custom epoch definition. 
	\end{itemize}

 \item Parameters (see Section~\ref{sec:params}).
\end{itemize}

The parameters are set via the graphical user interface. The time series 
$x(t)$ and a custom epoch definition can be loaded from comma-separated values
(CSV) and Matlab (.mat) files.

\subsection{Comma-Separated-Values File Format}
\label{sec:csv_input}

Comma Separated Values (CSV) files are human-readable text files for storing tabular data. 
The columns are separated by commas and each line of the file corresponds to a row. 
Lines starting with a hash (\texttt{\#}) are ignored. If the data has more rows than columns,
then each row will be interpreted as a time point and each column as a channel. Otherwise,
the format is assumed to be $\text{channels} \times \text{time}$. See 
Figure~\ref{fig:ex_timeseries} for an example.

\begin{figure}[h]
\centering
\begin{boxedminipage}{10cm}
\begin{verbatim}
# 2ch recording VPzj Oct 30th
-0.18671,0.11393
0.72579,1.0668
-0.58832,0.059281
2.1832,-0.095648
-0.1364,-0.83235
\end{verbatim}
\end{boxedminipage}
\caption{
Timeseries in CSV format with two channels and five time points. 
The first line is a comment for documentation purposes.
\label{fig:ex_timeseries}
}
\end{figure}

A custom segmentation of the time series into epochs can be specified by means of 
a separate CSV file, which must have the same number of rows as the time series 
and one column. The entries correspond to the index (starting with 1) of the 
epoch that a time point belongs to. Figure~\ref{fig:ex_segmentation} shows 
an example for a segmentation of the time series in Figure~\ref{fig:ex_timeseries}.

\begin{figure}[h]
\centering
\begin{boxedminipage}{10cm}
\begin{verbatim}
# Epochs for recording VPzj Oct 30th
1
1
1
2
2
\end{verbatim}
\end{boxedminipage}
\caption{
Custom epoch definition for the time series shown in Figure~\ref{fig:ex_timeseries}. 
The first three time points belong to the same epoch and the last two time points form the
second epoch.
\label{fig:ex_segmentation}
}
\end{figure}

\subsection{Matlab File Format}

In the Matlab file format, the time series must be contained in a variable
called \texttt{X}. If \texttt{X} has more rows than columns, then each row
will be interpreted as a time point and each column as a channel. Otherwise,
the format is assumed to be $\text{channels} \times \text{time}$. 

\begin{figure}[h]
\centering
\begin{boxedminipage}{10cm}
\begin{verbatim}
>> X

X =

   -0.4326   -0.1867
   -1.6656    0.7258
    0.1253   -0.5883
    0.2877    2.1832
   -1.1465   -0.1364

\end{verbatim}
\end{boxedminipage}
\caption{Timeseries in Matlab with two channels and five time points.
\label{fig:ex_matlabts}
}
\end{figure}

If \texttt{X} is a cell array, then the elements
are interpreted as epochs where each epoch must have the same number 
of channels, Figure~\ref{fig:ex_matlabsegs} shows
an example.

\begin{figure}[h]
\centering
\begin{boxedminipage}{10cm}
\begin{verbatim}
>> X

X = 

    [100x2 double]    [100x2 double]    [80x2 double]
\end{verbatim}
\end{boxedminipage}
\caption{Timeseries in Matlab with custom epoch definition. The time series is split
into three epochs where the first two epochs contain 100 samples each and the third
epoch consists of 80 samples.
\label{fig:ex_matlabsegs}
}
\end{figure}

\subsection{Parameters of SSA}
\label{sec:params}

The SSA algorithm has the following parameters which are set via the graphical user
interface.

\paragraph{Number of stationary sources} The number of stationary sources $d$ to be 
found in the time series. For the minimum number of epochs required to avoid spurious
stationary directions depends on $d$, see Section~\ref{sec:determinacy}. 

\paragraph{Number of restarts} The number of times the optimization procedure
should be repeated with different random initialization in order to avoid local minima.
The final result is the decomposition which attained the smallest objective function
value.

\paragraph{Number of equally-sized epochs} The number of equally sized epochs $N$ that 
the time series should be split into, if the user did not supply a custom segmentation.
The minimum number of epochs required to avoid spurious
stationary directions depends on the number of stationary sources $d$, see Section~\ref{sec:determinacy}. 

\paragraph{Which moments of the sources should be considered} Depending on the application domain, changes in mean or covariance matrix either do not occur or are not relevant. The user can therefore select whether non-stationarities in the mean, covariance matrix or both should be considered. The default is that both moments are taken into account.

\subsubsection*{Determinacy of the solution}
\label{sec:determinacy}

If the number of epochs $N$ is too small in relation to the number of 
non-stationary sources $D-d$, there may exist spurious stationary components that render
the solution non-identifiable, i.e. components which appear stationary on the
limited amount of observed data but which are in fact non-stationry.
Informally speaking, spurious stationary components occur when the amount of 
observed variation in the distributions (i.e. the number of epochs) is insufficient 
to eliminate seemingly stationary directions in the non-stationary subspace. 

It can be shown \cite{PRL:SSA:2009} that there exist no spurious stationary directions 
if the number of distinct epochs is larger than 
\begin{equation*}
	N > \frac{D-d}{2} + 2.
\end{equation*}
When only changes in one moment are considered, more
distinct epochs are needed for guaranteed determinacy,
\begin{equation*}
	N > D-d + 1 .
\end{equation*}
The SSA~Toolbox issues a warning if this condition is violated.

\section{Output}

The result of the SSA algorithm is an estimated demixing matrix $\hat{A}^{-1}$
(see Equation~\ref{eq:est_demixing}). From this, the following output
is generated by the SSA~Toolbox (assuming the output format $\text{channels} \times \text{time}$):
\begin{itemize}
	\item the estimated projection to the stationary sources $\hat{P}^\s \in \R^{d \times D}$, 
				see Equation~\ref{eq:est_demixing}
						
	\item the estimated projection to the non-stationary sources $\hat{P}^\n \in \R^{ (D-d) \times D}$, 
		see Equation~\ref{eq:est_demixing}

	\item the estimated basis of the stationary subspace $\hat{A}^\s \in \R^{D \times d}$, 
		see Equation~\ref{eq:est_mixing}
	
	\item the estimated basis of the non-stationary subspace $\hat{A}^\n \in \R^{D \times (D-d)}$,
		see Equation~\ref{eq:est_mixing}

	\item the estimated stationary sources $\hat{\mathbf s}^\s(t) = \hat{P}^\s {\mathbf x}(t)$, 
		see Equation~\ref{eq:est_s_sources}
	
	\item the estimated non-stationary sources $\hat{\mathbf s}^\n(t) = \hat{P}^\n {\mathbf x}(t) $,
		see Equation~\ref{eq:est_n_sources}
	
\end{itemize}
These six matrices resp.~time series can either be saved in individual CSV files 
(see Section~\ref{sec:csv_input} for a general description of the format) or to a single
Matlab file (.mat). 

\subsection{Matlab File Format}

The output of the SSA Toolbox can be written to a single Matlab file which contains
a structure \texttt{ssa\_results} where the attributes correspond to the six outputs. 
See Figure~\ref{fig:ex_matlab_res} for an example.

\begin{figure}[h]
\centering
\begin{boxedminipage}{12cm}
\begin{verbatim}
>> ssa_results = 

         est_Ps: [3x6 double]
         est_Pn: [3x6 double]
         est_As: [6x3 double]
         est_An: [6x3 double]
      est_s_src: [3x10000 double]
      est_n_src: [3x10000 double]
     parameters: [1x1 struct]
    description: 'SSA results (Thu Jun 10 12:58:23 CEST 2010)'
\end{verbatim}
\end{boxedminipage}
\caption{
Matlab result structure of the SSA Toolbox. Five dimensional data (500 samples) 
are decomposed into $d = 3$ stationary sources and $D-d = 2$ non-stationary sources.
\label{fig:ex_matlab_res}
}
\end{figure}

\section{Support}

If you have any question, bug report or want to discuss the application of SSA
to a specific problem, please join the SSA mailing list:
\begin{center}
        \url{http://groups.google.com/group/ssa-list}
\end{center}
You can find further
information on the official SSA homepage: 
\begin{center}
	\url{http://www.stationary-subspace-analysis.org}
\end{center}

\bibliographystyle{plain}
\bibliography{ssa.bib}

\end{document}
